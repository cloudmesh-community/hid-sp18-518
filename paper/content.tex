% status: 0
% chapter: TBD

\title{AWS CloudTrail}

\author{Michael Robinson}
\affiliation{%
  \institution{Indiana University}
  \country{USA}}
\email{micbrobi@umail.iu.edu}


% The default list of authors is too long for headers}
\renewcommand{\shortauthors}{M. Robinson}


\begin{abstract}
The AWS CloudTrail service~\cite{hid-sp18-518-CloudTrail} is an activity 
recording service provided by Amazon Web Services. The service allows you to 
track the history of account usage for your AWS instances. The service is not 
on by default yet when configured, it will record all API calls from all 
sources like the console, CLE, SDKs or CloudFormation. The data is written 
into an S3 bucket via JSON and would include attributes lik user, IP address, 
timestamp and the action the user took.
\end{abstract}

\keywords{hid-sp18-518, CloudTrail}

\maketitle

\section{Introduction}

Your API is finally working. After weeks of learning Docker, AWS Lambda and 
other Cloud technologies, you feel the pride of seeing your code finally 
providing a service. And that service is on the Internet! Yet, a few days 
later, you are notified by AWS that your AWS bill is about to reach the cap 
you set for it. That obviously can’t be right. As you log into your AWS 
account to check on what is going on, you notice containers running that you 
don’t recognize. How is this possible? Who did this? That is where 
CloudTrail~\cite{hid-sp18-518-CloudTrail} can help.

As cloud technologies are quickly adopted, one of the common mistakes is 
skipping past features related to security. To cover the aspect of
logging and monitoring, both features that are critical to ensuring 
appropriate use, and the intent is to help build understanding of how these 
features can help you keep your system uptime high, your bills low and provide 
you certainty that you know when changes occur.

CloudTrail is a scalable, extensible and simplied logging service available 
in AWS to be used to log all actions taken by multiple aspects of interacting
with AWS. In AWS, one of the primary means of providing logging is through AWS 
CloudTrail. Amazon explains it as a service that enables governance, 
compliance, operational auditing, and risk auditing of your AWS 
account~\cite{hid-sp18-518-CloudTrail-user-guide}. To simplify, AWS provides 
a logging service that can be used to meet logging requirements, either ones
you design personally or requirements that are directed by a higher entity 
such as government or a regulatory body such as PCI-DSS or HIPAA.

\section{How it works}

CloudTrail will record activity, based on your configuration, into an 
AWS S3 bucket~\cite{hid-sp18-518-CloudTrail-log-example}.
The JSON-formatted log files are compressed before they are written into an 
S3 bucket. The convention for storing the files consists of two 
parts, path and filename. The path will follow a convention of 
account/region/date and each day will have its own path.  The filename of the 
compressed file follows a convention of account/region/timestamp. 

You may have noticed that the default path of account to region may not work 
well for a multi-account, multi-region AWS instance. There are options to 
define logging for multi-region where the configuration can be applied to all 
known existing regions and even configured to include new ones when they come 
online by default~\cite{hid-sp18-518-CloudTrail-global-events}. Additionally, 
for multi-user situations, you can leverage AWS Identity and Access 
Management~\cite{hid-sp18-518-CloudTrail-IAM} to set global logging for all 
accounts as well.

To manage CloudTrail, you can use the AWS command line interface, RESTful APIs
or a software development kit for Python, Java or multiple other languages. 
These methods will allow you to make view the existing logs, which are natively 
built in for you for the first 90 days. The management interfaces also allow
you configure which S3 bucket to use, prefixes you may wish for organization, 
encryption keys to be used as well as more complex features like custom event
notifications to solutions like email, Slack channels or a text message.

To ensure no one can intentionally or accidentally delete your CloudTrail logs,
it is important to leverage features in Identity and Access Management. With 
Identity and Access Management, you can grant access per user on who is 
authorized to read logs stored in the CloudTrail S3 bucket 
or who can make changes to the CloudTrail configuration. It is also recommended 
to leverage features like Multi-Factor Authentication, which can be used to 
ensure that a request to delete an S3 bucket for CloudTrail logs is 
legitimate~\cite{hid-sp18-518-CloudTrail-user-guide}. 

Another aspect where IAM can help you with CloudTrail is ensuring the logs are 
owned by a service account. Access requests to the log data can be controlled 
by only authorizing access by log analysis and correlation service accounts and
 a human would never had direct access to the logs stored in S3. This helps 
ensure access control is not lost if your organizational use of AWS IAM 
continues to grow. 

Now that you have limited who could accidentally delete your logs, how do you 
ensure someone doesn’t just tamper with the logs? That is where the CloudTrail 
log integrity solution can help. The process will create hashes for the current 
log metadata and appends some information from the previous log archive. That 
way you can walk back the integrity of your logs and quickly identify which log
 was tampered with. The log validation service also provides you a tool to 
assist with validation and you just need to provide it a time range to 
validate~\cite{hid-sp18-518-CloudTrail-user-guide}.

Another best practice is to use the AWS Quick Setup to reduce error caused by 
manual implementation. The command create-subscription will automate the 
creation of an S3 bucket and will begin the login. An even better approach is 
to use features like AWS CloudFormation to automate CloudTrail configuration 
for every EC2 instance.

\section{How to view the logs}

To ensure no one can intentionally or accidentally delete your CloudTrail logs,
 it is important to leverage features in IAM. With IAM, you can grant access 
per user on who is authorized to read logs stored in the CloudTrail S3 bucket 
or who can make changes to the CloudTrail configuration. It is also recommended 
to leverage features like Multi-Factor Authentication, which can be used to 
ensure that a request to delete an S3 bucket for CloudTrail logs is legitimate.
 
An aspect where IAM can help you with CloudTrail is ensuring the logs are owned 
by a service account. Access requests to the log data can be controlled by only 
authorizing access by log analysis and correlation service accounts and a human
 would never had direct access to the logs stored in S3. This helps ensure 
access control is not lost if your organization’s use of AWS IAM continues to 
grow. 

Now that you have limited who could accidentally delete your logs, how do you 
ensure someone doesn’t just tamper with the logs? That is where the CloudTrail 
log integrity solution can help. The process will create hashes for the current 
log metadata and appends some information from the previous log archive. That 
way you can walk back the integrity of your logs and quickly identify which log 
was tampered with. The log validation service also provides you a tool to
 assist with validation and you just need to provide it a time range to 
validate~\cite{hid-sp18-518-CloudTrail-log-sharing}.

Another best practice is to use the AWS Quick Setup to reduce error caused by 
manual implementation. The command create-subscription will automate the 
creation of an S3 bucket and will begin the login. An even better approach is 
to use features like AWS CloudFormation to automate CloudTrail configuration 
for every EC2 instance.

\section{Conclusion}

This paper covered the AWS feature of CloudTrail which is a log event 
generating service. CloudTrail is used to monitor high-level access to your 
cloud infrastructure and also generates an audit trail to identify when changes 
occurred. Once configured, you can use other features to leverage CloudTrail 
logs so you can know immediately when something has changed. The power of AWS 
to automate and leverage templates is a great opportunity to build even faster, 
and even build security solutions even faster.

\begin{acks}

The author would like to thank Dr.~Gregor~von~Laszewski for his support and 
suggestions in writing this paper.

\end{acks}

\bibliographystyle{ACM-Reference-Format}
\bibliography{report} 

