% status: 0
% chapter: Virtual Machines

\title{RESTful API service to enable ease of portability of virtual
machines leveraging libcloud}


\author{Harshad Pitkar}
\affiliation{%
  \institution{Indiana University}
  \streetaddress{Smith Research Center}
  \city{Bloomington} 
  \state{IN} 
  \postcode{47408}
  \country{USA}}
\email{hpitkar@iu.edu}

\author{Sushant Athaley}
\affiliation{%
  \institution{Indiana University}
  \streetaddress{Smith Research Center}
  \city{Bloomington} 
  \state{IN} 
  \postcode{47408}
  \country{USA}}
\email{sathaley@iu.edu}

\author{Michael Robinson}
\affiliation{%
  \institution{Indiana University}
  \streetaddress{Smith Research Center}
  \city{Bloomington} 
  \state{IN} 
  \postcode{47408}
  \country{USA}}
\email{micbrobi@iu.edu}


% The default list of authors is too long for headers}
\renewcommand{\shortauthors}{H. Pitkar, S. Athaley, M. Robinson}


\begin{abstract}
Our research will measure how the portability of an application is
impacted by decisions made early in the software lifecycle when native
cloud provider APIs are used. We will demonstrate how efforts to
leverage scalable, reproducable solutions like boto and libcloud are
advantageous. We will derive reproducable code using Swagger and Python
to show how libraries like boto and libcloud can be used to migrate an
application from one cloud provider to another with a measurable
reduction in human error and with less time to execute. Additionally, we
will leverage Swagger to develop RESTful APIs to further improve the
gains. We will capture the results of the research, share the derived
code and conclude how the research can be applied to existing and new
development efforts intending to leverage cloud providers. 
\end{abstract}

\keywords{hid-sp18-518, hid-sp18-517, hid-sp18-402, libcloud}


\maketitle

\section{Introduction}\label{introduction}

Cloud portability is a growing area of research due to the increased
profiliferation of cloud providers~\cite{hid-sp18-518-Cloud-Council}. 
Each provider has unique APIs and tools to their cloud environments 
which can disincentivize portability as it influences a consumer to 
stay with their existing solution provider. Efforts to standardize 
cloud portability like TOSCA have made progress yet participation by 
cloud providers is constrained due to the competitive nature in the 
space. Each cloud provider is looking to retain their userbase and 
there is also a desire by each provider to become the de facto 
standard by being the market leader. To fill the gap, solutions like 
Apache libcloud and boto have delivered an abstraction solution to 
developers to design applications that are easy to port.

Developers are already confronted with a lack of transparency on which
cloud provider is optimal for the long-term sustainability of their
application. Additionally, attempts to abstract away from cloud
providers are helpful yet their non-standardization still potentially
locks you into the solutions provided by Apache or communities like
boto. We will deliver a continuation of that abstraction concept with an
accepted standard, REST, to extend libcloud and boto. By leveraging
REST, we intend to introduce a standardized implementation that
leverages cloud portability libraries to manage the diversity of cloud
applications. A high-level abstract of our final concept is below.

\TODO{THis is Figure~\ref{F:arch}} and its not in pdf.

\begin{figure}[!ht]
  \centering
  \includegraphics[width=\columnwidth]{images/proj-arch.png}
  \caption{Project Architecture}\label{F:arch}
\end{figure}


\section{Scope of work}\label{scope-of-work}

\begin{itemize}
\item
  Value Hypothesis
\item
  Cycle time metrics on porting a solution without boto/libcloud
\item
  Lead time metrics on use of RESTful APIs compared to command line
  usage
\item
  RESTful API

  \begin{itemize}
    \item
    Swagger development to design/build API services for UI
  \item
    Python development to leverage boto/libcloud
  \item
    Evidence of success porting a solution from/to AWS, Azure, Google
    Cloud
  \end{itemize}
\item
  Comparison of porting solution manually to libcloud/boto by command
  line
\item
  Comparison of porting solution with libcloud/boto by commandline to
  RESTful API solution
\item
  Conclusion
\end{itemize}

\hypertarget{special-consideration-to-project-format}{%
\section{Special Consideration to Project
Format}\label{special-consideration-to-project-format}}

\begin{itemize}
\item
  Swagger API documentation
\item
  Comparison of command line tools to Python libraries libcloud and boto
\end{itemize}

\begin{acks}

  The authors would like to thank Dr.~Gregor~von~Laszewski for his
  support and suggestions to write this paper.

\end{acks}

\bibliographystyle{ACM-Reference-Format}
\bibliography{report} 

