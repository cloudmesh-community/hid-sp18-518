% status: 0
% chapter: Virtual Machines
% 
\title{RESTful API service to enable ease of portability of virtual
machines leveraging libcloud}


\author{Harshad Pitkar}
\affiliation{%
  \institution{Indiana University}
  \streetaddress{Smith Research Center}
  \city{Bloomington} 
  \state{IN} 
  \postcode{47408}
  \country{USA}}
\email{hpitkar@iu.edu}

\author{Sushant Athaley}
\affiliation{%
  \institution{Indiana University}
  \streetaddress{Smith Research Center}
  \city{Bloomington} 
  \state{IN} 
  \postcode{47408}
  \country{USA}}
\email{sathaley@iu.edu}

\author{Michael Robinson}
\affiliation{%
  \institution{Indiana University}
  \streetaddress{Smith Research Center}
  \city{Bloomington} 
  \state{IN} 
  \postcode{47408}
  \country{USA}}
\email{micbrobi@iu.edu}


% The default list of authors is too long for headers}
\renewcommand{\shortauthors}{H. Pitkar, S. Athaley, M. Robinson}


\begin{abstract}
Our research will measure how the portability of an application is impacted by decisions 
made early in the software lifecycle when native cloud provider APIs are used. We will 
demonstrate how efforts to leverage scalable, reproducable solutions like boto and 
libcloud are advantageous. We will derive reproducable code using Swagger and Python
to show how libraries like boto and libcloud can be used to migrate an application from 
one cloud provider to another with a measurable reduction in human error and with less 
time to execute. Additionally, we will leverage Swagger to develop RESTful APIs to 
further improve the gains. We will capture the results of the research, share the derived 
code and conclude how the research can be applied to existing and new development efforts 
intending to leverage cloud providers. 
\end{abstract}

\keywords{hid-sp18-518, hid-sp18-517, hid-sp18-402, libcloud}


\maketitle

\section{Introduction}\label{introduction}

Cloud portability is a growing area of research due to the increased profiliferation of 
cloud providers~\cite{hid-sp18-518-Cloud-Council}. Each provider has unique APIs and 
tools to their cloud environments  which can disincentivize portability as it influences 
a consumer to  stay with their existing solution provider. Efforts to standardize 
cloud portability like TOSCA have made progress yet participation by cloud providers is 
constrained due to the competitive nature in the  space. Each cloud provider is looking 
to retain their userbase and there is also a desire by each provider to become the de 
facto  standard by being the market leader. To fill the gap, solutions like Apache 
libcloud and boto have delivered an abstraction solution to developers to design 
applications that are easy to port.

Developers are already confronted with a lack of transparency on which cloud provider is 
optimal for the long-term sustainability of their application. Additionally, attempts to 
abstract away from cloud providers are helpful yet their non-standardization still 
potentially locks you into the solutions provided by Apache or communities like boto. We 
will deliver a continuation of that abstraction concept with an accepted standard, REST, 
to extend libcloud and boto. By leveraging REST, we intend to introduce a standardized 
implementation that leverages cloud portability libraries to manage the diversity of 
cloud applications. A high-level abstract of the concept is shown in Figure~\ref{F:arch}.

\begin{figure}[!ht]
  \centering
  \includegraphics[width=\columnwidth]{images/proj-arch.pdf}
  \caption{Project Architecture}\label{F:arch}
\end{figure}


\section{Literature Review}

The National Institute of Standards and Technology defines cloud  portability as ``data 
that can be moved from one cloud system to another and that  applications can be ported 
and run on different  cloud systems at an  acceptable cost.''~\cite{hid-sp18-518-NIST-291} 
The concept of portability can be extended to encompass the full application stack from
the web service to the underlying hardware itself. Portability can also simply mean the 
ability to ensure high availability where you only are looking to protect against one 
cloud provider being a single point of faiure.

In NIST Special Publication 500-293, the United States governement has defined a strategic
roadmap that includes ten formal recommendations for all cloud usage. Out of the ten 
requirements, eight of them reference portability and interoperability. The Standards 
Acceleration to Jumpstart the Adoption of Cloud Computing (SAJACC) is an initiative under
the guidance of NIST 500-293 that is to define ``qualitative testing of specifications 
against interoperability, security, and portability requirements.''~\cite{hid-sp18-518-NIS
T-293}

To define portability further, we have to differentiate the tiers of cloud service and 
where portability may be needed. Cloud providers have generally grouped service into the
following four types.

\begin{itemize}
\item
  Infrastructure as a Service - IaaS
\item
  Platform as a Service - PaaS
\item
  Software as a Service - SaaS
\item
  Functions as a Service - FaaS
\end{itemize}

Each grouping has dependencies that can make portability more difficult. For example, AWS
Lamba, which is a FaaS solution, is highly specialized and the APIs in use are specific to
that vendor. While solutions like libcloud and boto attempt to include all providers, the 
speed of the market makes it challenging for portability libraries to include the latest 
and great cloud provider offerings.~\cite{hid-sp18-518-LibCloud} Another dependency is 
the complexity of what needs to be ported. IaaS is the closest cloud offering to 
bare-metal and dependencies for hardware-specific requirements are not a consideration 
for most portability offerings. As the adoption of containers and functions increases, 
legacy implementations of cloud solutions that leveraged IaaS will be more difficult to 
port over. 

The work by the Irish Centre for Cloud Computing provided a ``qualitative  comparative  
of current  open-source  IaaS frameworks'' which is in contrast to vendor offerings which
tend to lack in portability~\cite{hid-sp18-518-Comp-study}. The study was limited to the
five top open-source providers of IaaS and the derived outcome of the comparison was a 
breakdown over twenty categories that included portability to vendor IaaS. The summary
was that each solution is tailored towards a specific need and while portability is
possible, there are other challenges that are considered when choosing when and which
of many cloud providers you will end up using.

The research by Kostoska, Gusev and Ristov further highlights the challenges with 
portability, open-source and standards. The researchers were ``motivated by several open 
research questions about cloud solutions, such as how to wisely choose a cloud host for 
services and how to change the cloud provider in an easy manner.''\cite{hid-sp18-518-
Kostoska-Gusev-Ristov} The paper stipulates that not only is it difficult to choose
which cloud provider to use but that the community of cloud providers and what 
differentiates them continues to grow. The work concludes that no standard exists and 
their own efforts are only to ``offer a possibility for a documented service exchange.''

An interesting example of where multiple private sector providers can ensure 
interoperability is illustrated by Wired journalist Joe Weinman. In his writings, he
expresses how air travel is easy for a consumer to determine where to fly out of, how
to get through security, and to have confidence their bags will arrive. The history of
aviation is one of consolidation and resistance to standards yet the market ultimately 
did accept some level of standardization.~\cite{hid-sp18-518-Wired} Weinman concludes 
with ``The Internet took decades to go from a vision of packet switching to where it is 
today.  Between the IEEE, industry, and academia, one can hope that the vision of an 
Intercloud is now getting the attention it deserves.''

\section{Methodology}

Python libraries exist now that help you abstract your project from  your cloud provider. 
This may be important if you think you may need to use multiple cloud providers or if you 
want to ensure you take steps to avoid provider lock-in. There are multiple providers in 
this space and one of them is Apache Libcloud~\cite{hid-sp18-518-LibCloud} To put it 
simply, this library allows you to code simple resource  management that is independent 
of cloud provider specific API calls.

\begin{itemize}
\item
  Value Hypothesis
\item
  Cycle time metrics on porting a solution without boto/libcloud
\item
  Lead time metrics on use of RESTful APIs compared to command line
  usage
\item
  RESTful API

  \begin{itemize}
    \item
    Swagger development to design/build API services for UI
  \item
    Python development to leverage boto/libcloud
  \item
    Evidence of success porting a solution from/to AWS, Azure, Google
    Cloud
  \end{itemize}
\item
  Comparison of porting solution manually to libcloud/boto by command
  line
\item
  Comparison of porting solution with libcloud/boto by commandline to
  RESTful API solution
\item
  Conclusion
\end{itemize}

\section{Results and Discussion}

\begin{itemize}
\item
  Swagger API documentation
\item
  Comparison of command line tools to Python libraries libcloud and boto
\end{itemize}

\section{Summary}

\section{Scope of work}\label{scope-of-work}

\begin{acks}

  The authors would like to thank Dr.~Gregor~von~Laszewski for his
  support and suggestions to write this paper.

\end{acks}

\bibliographystyle{ACM-Reference-Format}
\bibliography{report} 

